\chapter{Introduction}

The goal of this document is to demonstrate a design for a feed handler that allows it to be run as both a standalone executable and as a shared library that can be loaded into a q processes. The architecture for the feed handler should share as much code as possible in order to increase maintainability. In order to facilitate this architecture, all processing of the data from the feed will take place in a background thread so that the main thread of the q process is still responsive. This use of threading can introduce its own problems when you need to marshal data between the feed handler and the main q process, which we provide a solution for in this paper. The solution provided is just one of the possible ways to structure a feed handler and may or may
not be the best solution depending on the circumstances.

To begin, the document will also quickly cover some of the basics of compiling and linking
the different types of binaries (executables and libraries) on each platform and explain
how to load and use the libraries within a q process. If you don't have a kdb+tick system
already set up to test your feed handler, you can download the \textbf{AquaQ TorQ}\footnote{https://github.com/AquaQAnalytics/TorQ} and the
\textbf{AquaQ TorQ Starter Pack}\footnote{https://github.com/AquaQAnalytics/TorQ-Finance-Starter-Pack}. TorQ should allow you to get up and
running with a production grade kdb+tick setup as soon as possible.





