\chapter{Requirements and Resources}

In order to follow this document, you should be comfortable with the following topics:

\begin{description}
	\item[C/C++] It is assumed that the reader will have some intermediate knowledge
	of C or C++ (e.g. regarding pointers/references).
	
	\item[The KDB+/C API] The basics of the C api will generally not be covered in 
	the document. The document will cover some of the issues that are more relevant
	to writing a feed handler.

	\item[IPC in kdb+] You should be able to open connections between multiple q
	processes and send messages between them. Knowledge of the difference between
	synchronous and asynchronous communication is also assumed.

	\item[kdb+tick] You should be familiar with the core components of a basic
	kdb+tick capture system (e.g. tickerplant, rdb, hdb).
\end{description}

You will need the following tools in order to compile some of the code examples:

\begin{description}
	\item[Visual Studio 2010+] If you wish to compile the 64 bit versions of the
	binaries, you will need to have the Ultimate Edition of Visual Studio, or the
	2013 Community Edition.
	
	\item[KDB+ 3.2] You can download the latest version of kdb+ from the Kx System
	website. The code in this document has been tested on version 3.2 (released 2014.12.05).
\end{description}

Some additional documentation that may be useful when following this document:

\begin{description}
	\item[\href{http://www.aquaq.co.uk/resources}{AquaQ Resources}] Contains various resources on interfacing
	with kdb+ via C, TCP and other kdb+ topics that maybe be useful.
	
	\item[\href{http://code.kx.com/wiki/Cookbook/InterfacingWithC}{Interfacing With C (Kx Wiki)}] This section
	of the Kx wiki explains the basics of interfacing with kdb+ via C.

	\item[\href{http://code.kx.com/wiki/Startingkdbplus/tick}{Starting kdb+}] Describes the basics of a typical
	kdb+tick capture system.
\end{description}